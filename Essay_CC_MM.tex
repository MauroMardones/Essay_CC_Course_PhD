% Options for packages loaded elsewhere
\PassOptionsToPackage{unicode}{hyperref}
\PassOptionsToPackage{hyphens}{url}
\PassOptionsToPackage{dvipsnames,svgnames*,x11names*}{xcolor}
%
\documentclass[
]{article}
\usepackage{lmodern}
\usepackage{amssymb,amsmath}
\usepackage{ifxetex,ifluatex}
\ifnum 0\ifxetex 1\fi\ifluatex 1\fi=0 % if pdftex
  \usepackage[T1]{fontenc}
  \usepackage[utf8]{inputenc}
  \usepackage{textcomp} % provide euro and other symbols
\else % if luatex or xetex
  \usepackage{unicode-math}
  \defaultfontfeatures{Scale=MatchLowercase}
  \defaultfontfeatures[\rmfamily]{Ligatures=TeX,Scale=1}
\fi
% Use upquote if available, for straight quotes in verbatim environments
\IfFileExists{upquote.sty}{\usepackage{upquote}}{}
\IfFileExists{microtype.sty}{% use microtype if available
  \usepackage[]{microtype}
  \UseMicrotypeSet[protrusion]{basicmath} % disable protrusion for tt fonts
}{}
\usepackage{xcolor}
\IfFileExists{xurl.sty}{\usepackage{xurl}}{} % add URL line breaks if available
\IfFileExists{bookmark.sty}{\usepackage{bookmark}}{\usepackage{hyperref}}
\hypersetup{
  colorlinks=true,
  linkcolor=blue,
  filecolor=Maroon,
  citecolor=Blue,
  urlcolor=Blue,
  pdfcreator={LaTeX via pandoc}}
\urlstyle{same} % disable monospaced font for URLs
\usepackage[margin=2cm]{geometry}
\usepackage{graphicx}
\makeatletter
\def\maxwidth{\ifdim\Gin@nat@width>\linewidth\linewidth\else\Gin@nat@width\fi}
\def\maxheight{\ifdim\Gin@nat@height>\textheight\textheight\else\Gin@nat@height\fi}
\makeatother
% Scale images if necessary, so that they will not overflow the page
% margins by default, and it is still possible to overwrite the defaults
% using explicit options in \includegraphics[width, height, ...]{}
\setkeys{Gin}{width=\maxwidth,height=\maxheight,keepaspectratio}
% Set default figure placement to htbp
\makeatletter
\def\fps@figure{htbp}
\makeatother
\setlength{\emergencystretch}{3em} % prevent overfull lines
\providecommand{\tightlist}{%
  \setlength{\itemsep}{0pt}\setlength{\parskip}{0pt}}
\setcounter{secnumdepth}{-\maxdimen} % remove section numbering
\newlength{\cslhangindent}
\setlength{\cslhangindent}{1.5em}
\newenvironment{cslreferences}%
  {\setlength{\parindent}{0pt}%
  \everypar{\setlength{\hangindent}{\cslhangindent}}\ignorespaces}%
  {\par}

\title{\includegraphics[width=3cm,height=\textheight]{logoUMAG.jpg}}
\author{}
\date{\vspace{-2.5em}}

\begin{document}
\maketitle


\pagenumbering{gobble}

%\begin{titlepage}
\begin{flushleft}
\Large{\textbf{Ensayo Científico}}\\
\vspace*{2\baselineskip}
\LARGE{\textbf{Impactos del Cambio Climático en los patrones de distribución de poblaciones marinas en ecosistemas Antárticos y Sub-Antárticos}}\\
\vspace*{5\baselineskip}
\Large{Curso Cambio Climático}\\
\Large{Semestre-1 2021 }\\
\vspace*{4\baselineskip}
\end{flushleft}
\begin{flushright}
\large{\textbf{Mauricio Mardones Inostroza}}\\
\vspace*{2\baselineskip}
\normalsize{Programa de Doctorado Ciencias Antárticas y Sub-Antárticas}\\
\vspace*{1\baselineskip}
\normalsize{Universidad de Magallanes, Chile}\\
\vspace*{2\baselineskip}
\normalsize{\textbf{Profesores}}\\
Dr. Rodrigo Villa-Martinez\\
Dr. Juan Carlos Aravena\\
\vspace*{1\baselineskip}
\normalsize{\textbf{Fecha}}\\
Mayo, 2021
\end{flushright}

% \end{titlepage}


\hypersetup{linkcolor = black}
\newpage
\pagenumbering{roman}
%\tableofcontents
%\addcontentsline{toc}{section}{\contentsname}

\newpage



\pagenumbering{arabic}
\hypersetup{linkcolor = blue}

\fontsize{12}{26}
\selectfont{}

\hypertarget{introduccion}{%
\subsection{INTRODUCCION}\label{introduccion}}

actualmente existen variados estudios que tratan de abordar las
respuestas de las especies a forzantes climáticas en un escenario de
cambios (Jacquet, Blood-Patterson, Brooks, \& Ainley,
\protect\hyperlink{ref-Jacquet2016a}{2016}; Lucey et al.,
\protect\hyperlink{ref-Lucey2021}{2021}; Plagányi et al.,
\protect\hyperlink{ref-Plaganyi2014}{2014}; Trathan,
\protect\hyperlink{ref-Trathan2017}{2017}). Existen multiples evidencias
que demuestran y magnifican el impacto del Camblìo Climático en
distintos aspectos en la poblaciones de especies marinas, ya sean estas;
mamóferos {[}cita{]}, peces {[}cita{]}, moluscos {[}cita{]} y diversos
organismos marinos que constituyen comunidades ecologicas en distintos
ecosistemas del planeta. Algunos autores proponen que los impactos del
cambio climatico, a traves de un calentamiento de las masas de agua,
gatillarìa cambios en los patrones de distribución espacial de los
organismos marinos, incluso, haciendolos migrar hacia altas latitudes.
En este sentido, es necesario preguntar como impactaran las condiciones
oceanograficas cambiantes en ecosistemas de altas latitudes, y com
responderan las especies marinas que allí habitan, o bien las que
llegaran a estas latitudes. En ese sentido, regiones polares denes ser
analizadas a la luz de la evidencia cientifica en este aspecto,
identificando los cambios ocurridos, asi como también, proyectar los
impoactos del cambio climatico en estas poblaciones (Pitman, Fearnbach,
\& Durban, \protect\hyperlink{ref-Pitman2018a}{2018}) y (Abrams et al.,
\protect\hyperlink{ref-Abrams2016a}{2016})

Este ensayo tiene como objetivo identificar que cambios se producen en
el ambiente marino a causa del cambio climático recirente, los cambios
de distribución de las poblaciones marinas, y por último, tambien
identificar escenarios futuros y su implcancia en la sociedad humana
(Melnychuk, Banobi, \& Hilborn,
\protect\hyperlink{ref-Melnychuk2014}{2014}; Rijnsdorp, Peck, Engelhard,
Möllmann, \& Pinnegar, \protect\hyperlink{ref-Rijnsdorp2009}{2009}).

\pagebreak

\hypertarget{cuerpo}{%
\subsection{CUERPO}\label{cuerpo}}

Antecedentes científicos que confirmen estos impactos

\pagebreak

\hypertarget{discusion}{%
\subsection{DISCUSION}\label{discusion}}

Opinion personal respecto a los anteceentes presentados y el prpblema
formulado

\pagebreak

\hypertarget{referencias}{%
\subsection*{REFERENCIAS}\label{referencias}}
\addcontentsline{toc}{subsection}{REFERENCIAS}

\hypertarget{refs}{}
\begin{cslreferences}
\leavevmode\hypertarget{ref-Abrams2016a}{}%
Abrams, P. A., Ainley, D. G., Blight, L. K., Dayton, P. K., Eastman, J.
T., \& Jacquet, J. L. (2016). Necessary elements of precautionary
management: implications for the Antarctic toothfish. \emph{Fish and
Fisheries}, \emph{17}(4), 1152--1174.
\url{https://doi.org/10.1111/faf.12162}

\leavevmode\hypertarget{ref-Jacquet2016a}{}%
Jacquet, J., Blood-Patterson, E., Brooks, C., \& Ainley, D. (2016).
'Rational use' in Antarctic waters. \emph{Marine Policy}, \emph{63},
28--34. \url{https://doi.org/10.1016/j.marpol.2015.09.031}

\leavevmode\hypertarget{ref-Lucey2021}{}%
Lucey, S. M., Aydin, K. Y., Gaichas, S. K., Cadrin, S. X., Fay, G.,
Fogarty, M. J., \& Punt, A. (2021). Evaluating fishery management
strategies using an ecosystem model as an operating model.
\emph{Fisheries Research}, \emph{234}(April 2020).
\url{https://doi.org/10.1016/j.fishres.2020.105780}

\leavevmode\hypertarget{ref-Melnychuk2014}{}%
Melnychuk, M. C., Banobi, J. A., \& Hilborn, R. (2014). The adaptive
capacity of fishery management systems for confronting climate change
impacts on marine populations. \emph{Reviews in Fish Biology and
Fisheries}, \emph{24}(2), 561--575.
\url{https://doi.org/10.1007/s11160-013-9307-9}

\leavevmode\hypertarget{ref-Pitman2018a}{}%
Pitman, R. L., Fearnbach, H., \& Durban, J. W. (2018). Abundance and
population status of Ross Sea killer whales (Orcinus orca, type C) in
McMurdo Sound, Antarctica: evidence for impact by commercial fishing?
\emph{Polar Biology}, \emph{41}(4), 781--792.
\url{https://doi.org/10.1007/s00300-017-2239-4}

\leavevmode\hypertarget{ref-Plaganyi2014}{}%
Plagányi, É. E., Punt, A. E., Hillary, R., Morello, E. B., Thébaud, O.,
Hutton, T., \ldots{} Rothlisberg, P. C. (2014). Multispecies fisheries
management and conservation: Tactical applications using models of
intermediate complexity. \emph{Fish and Fisheries}, \emph{15}(1), 1--22.
\url{https://doi.org/10.1111/j.1467-2979.2012.00488.x}

\leavevmode\hypertarget{ref-Rijnsdorp2009}{}%
Rijnsdorp, A. D., Peck, M. A., Engelhard, G. H., Möllmann, C., \&
Pinnegar, J. K. (2009). Resolving the effect of climate change on fish
populations. \emph{ICES Journal of Marine Science}, \emph{66}(7),
1570--1583. \url{https://doi.org/10.1093/icesjms/fsp056}

\leavevmode\hypertarget{ref-Trathan2017}{}%
Trathan, P. (2017). \emph{Managing the fishery for \{Antarctic\} krill:
\{A\} brief review of important environmental and management
considerations}. 5p.
\end{cslreferences}

\end{document}
